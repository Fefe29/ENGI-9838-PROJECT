\documentclass[12pt,a4paper]{article}
\usepackage{geometry}
\usepackage{amsmath}
\usepackage{graphicx}
\usepackage[hidelinks]{hyperref}
\usepackage{float}
\usepackage{bookmark}
\usepackage{array}
\usepackage{xcolor}
\usepackage{longtable}

\geometry{margin=1in}

\title{\textbf{Assignment 2: Report}}
\author{}
\date{}

\begin{document}

\maketitle

\section*{Group Members}
\vspace{-5pt}
\begin{itemize}
    \item \textbf{Mohsen IranianGhareshiran} - 202382341
    \item \textbf{Galilea Le Moullec} - 202415993
    \item \textbf{Félicien Moquet} - 202415994
    \item \textbf{Kateryna Nazarenko} - 202415995
\end{itemize}


\section{System Architecture and UML Diagrams}

\subsection{Class Diagram for Core Modules}
The Automated Online Test Monitoring System consists of multiple modules:
\begin{itemize}
\item \textbf{User Module}: Handles test-taker authentication and access management.
\item \textbf{Monitoring Module}: Utilizes AI-based facial recognition, gaze tracking, and motion detection.
\item \textbf{Anti-Spoofing Module}: Implements deepfake detection and liveness verification.
\item \textbf{Reporting Module}: Generates logs and alerts for test administrators.
\item \textbf{Integration Module}: Ensures compatibility with existing LMS such as Moodle or ExamSoft.
\end{itemize}

\subsection{Use Case Diagram for User Interactions}
The following actors interact with the system:
\begin{itemize}
\item \textbf{Test-taker}: Participates in the online exam while being monitored.
\item \textbf{Administrator}: Reviews reports and adjusts monitoring settings.
\item \textbf{System AI}: Performs real-time analysis and reporting.
\end{itemize}

Use case interactions include:
\begin{itemize}
\item Start Exam (Test-taker initiates a session)
\item Monitor Test (AI-driven real-time observation)
\item Detect Fraudulent Behavior (Anti-spoofing mechanisms activate upon suspicious activity)
\item Generate Report (System logs flagged behaviors and generates reports)
\item Administrator Review (Admin evaluates reports and decides on any actions)
\end{itemize}

\subsection{Sequence Diagram for Key Workflows}
\begin{enumerate}
\item Test-taker logs in and starts the exam.
\item AI-driven monitoring begins (facial recognition, motion tracking, gaze detection).
\item If suspicious behavior is detected, an alert is generated.
\item The system logs the behavior and updates the administrator dashboard.
\item The administrator reviews the report post-exam.
\end{enumerate}

\section{Test Plan and Suite}

\subsection{Unit Tests}
\begin{itemize}
\item \textbf{Facial Recognition Module}: Test AI accuracy in recognizing legitimate users.
\item \textbf{Anti-Spoofing Measures}: Validate that deepfake detection correctly flags fraudulent activities.
\item \textbf{Logging \& Reporting}: Ensure flagged incidents are stored accurately in the database.
\end{itemize}
\textbf{Tools:} JUnit for backend logic, OpenCV for AI validation.

\subsection{API Tests}
If microservices are used, API tests will ensure:
\begin{itemize}
\item Secure authentication for test-takers.
\item Efficient data exchange between LMS and monitoring system.
\item Accurate data retrieval for post-exam reviews.
\end{itemize}
\textbf{Tools:} Postman, Jest (for API verification)

\section{Justification of Design Choices}

\subsection{Transition to Microservices}
Our system is structured using a \textbf{microservices architecture} to enable modularity, scalability, and ease of integration with existing LMS. The following design choices were made:
\begin{itemize}
\item \textbf{Decoupled AI Processing}: The AI-driven monitoring module operates independently, reducing system overhead.
\item \textbf{Scalable Cloud Deployment}: The system is deployed on AWS/GCP, allowing horizontal scaling.
\item \textbf{Secure Data Handling}: Only AI-generated reports are stored to maintain privacy compliance.
\end{itemize}

\subsection{Why Microservices?}
\begin{itemize}
\item \textbf{Modularity}: Easier debugging and independent module updates.
\item \textbf{Scalability}: Allows handling thousands of concurrent test sessions.
\item \textbf{Interoperability}: Seamless integration with various LMS platforms.
\end{itemize}

\section{Team Contribution \& Documentation}

\begin{longtable}{|>{\raggedright\arraybackslash}p{4cm}|>{\raggedright\arraybackslash}p{10cm}|}
\hline
\textbf{Team Member} & \textbf{Contribution} \\ \hline
Mohsen IranianGhareshiran & Developed AI-based facial recognition and motion tracking \\ \hline
Galilea Le Moullec & Designed system architecture and microservices integration \\ \hline
Félicien Moquet & Worked on test plan, unit testing, and API verification \\ \hline
Kateryna Nazarenko & Developed reporting module and integration with LMS \\ \hline
\end{longtable}

\subsection{Summary of Contributions}
Each member contributed to both the conceptualization and implementation of the system. We collaborated on system design, ensuring alignment with the feasibility study.

\section{Conclusion}
This assignment outlines the core \textbf{design, architecture, and testing} framework for our \textbf{Automated Online Test Monitoring System}. By adopting \textbf{microservices architecture, AI-driven monitoring, and cloud deployment}, our system ensures \textbf{scalability, security, and reliability} in remote assessments. The proposed \textbf{testing strategy and design justification} further enhance the feasibility of a \textbf{cost-effective and privacy-conscious} solution for online proctoring.

\end{document}
